%%%%%%%%%%%%%%%%%%%%%%%%%%%%%%%%%%%%%%%%%12pt: grandezza carattere
                                        %a4paper: formato a4
                                        %openright: apre i capitoli a destra
                                        %twoside: serve per fare un
                                        %   documento fronteretro
                                        %report: stile tesi (oppure book)
\documentclass[12pt,a4paper,openright,twoside]{report}
%
%%%%%%%%%%%%%%%%%%%%%%%%%%%%%%%%%%%%%%%%%libreria per scrivere in italiano
\usepackage[italian]{babel}
%
%%%%%%%%%%%%%%%%%%%%%%%%%%%%%%%%%%%%%%%%%libreria per accettare i caratteri
                                        %   digitati da tastiera come è à
                                        %   si può usare anche
                                        %   \usepackage[T1]{fontenc}
                                        %   però con questa libreria
                                        %   il tempo di compilazione
                                        %   aumenta
\usepackage[utf8]{inputenc}
%
%%%%%%%%%%%%%%%%%%%%%%%%%%%%%%%%%%%%%%%%%libreria per impostare il documento
\usepackage{fancyhdr}
%
%%%%%%%%%%%%%%%%%%%%%%%%%%%%%%%%%%%%%%%%%libreria per avere l'indentazione
%%%%%%%%%%%%%%%%%%%%%%%%%%%%%%%%%%%%%%%%%   all'inizio dei capitoli, ...
\usepackage{indentfirst}

\usepackage{url}
%
%%%%%%%%%libreria per mostrare le etichette
%\usepackage{showkeys}
%
%%%%%%%%%%%%%%%%%%%%%%%%%%%%%%%%%%%%%%%%%libreria per inserire grafici
\usepackage{graphicx}
%
%%%%%%%%%%%%%%%%%%%%%%%%%%%%%%%%%%%%%%%%%libreria per utilizzare font
                                        %   particolari ad esempio
                                        %   \textsc{}
\usepackage{newlfont}

\usepackage{listings}
\lstset{breaklines=true
}
%
%%%%%%%%%%%%%%%%%%%%%%%%%%%%%%%%%%%%%%%%%librerie matematiche
\usepackage{amssymb}
\usepackage{amsmath}
\usepackage{latexsym}
\usepackage{amsthm}

%
\oddsidemargin=30pt \evensidemargin=20pt%impostano i margini
\hyphenation{sil-la-ba-zio-ne pa-ren-te-si}%serve per la sillabazione: tra parentesi 
					   %vanno inserite come nell'esempio le parole 
%					   %che latex non riesce a tagliare nel modo giusto andando a capo.

%
%%%%%%%%%%%%%%%%%%%%%%%%%%%%%%%%%%%%%%%%%comandi per l'impostazione
                                        %   della pagina, vedi il manuale
                                        %   della libreria fancyhdr
                                        %   per ulteriori delucidazioni
\pagestyle{fancy}\addtolength{\headwidth}{20pt}
\renewcommand{\chaptermark}[1]{\markboth{\thechapter.\ #1}{}}
\renewcommand{\sectionmark}[1]{\markright{\thesection \ #1}{}}
\rhead[\fancyplain{}{\bfseries\leftmark}]{\fancyplain{}{\bfseries\thepage}}
\cfoot{}
%%%%%%%%%%%%%%%%%%%%%%%%%%%%%%%%%%%%%%%%%
\linespread{1.3}                        %comando per impostare l'interlinea
%%%%%%%%%%%%%%%%%%%%%%%%%%%%%%%%%%%%%%%%%definisce nuovi comandi
%
\begin{document}
\begin{titlepage}                       %crea un ambiente libero da vincoli
                                        %   di margini e grandezza caratteri:
                                        %   si pu\`o modificare quello che si
                                        %   vuole, tanto fuori da questo
                                        %   ambiente tutto viene ristabilito
%
\thispagestyle{empty}                   %elimina il numero della pagina
\topmargin=6.5cm                        %imposta il margina superiore a 6.5cm
\raggedleft                             %incolonna la scrittura a destra
\large                                  %aumenta la grandezza del carattere
                                        %   a 14pt
                             
%
%%%%%%%%%%%%%%%%%%%%%%%%%%%%%%%%%%%%%%%%
\clearpage{\pagestyle{empty}\cleardoublepage}%non numera l'ultima pagina sinistra
\end{titlepage}
\pagenumbering{roman}                   %serve per mettere i numeri romani
\chapter{Lavori Correlati}                 %crea l'introduzione (un capitolo

                                        %   non numerato)
%%%%%%%%%%%%%%%%%%%%%%%%%%%%%%%%%%%%%%%%%imposta l'intestazione di pagina
\rhead[\fancyplain{}{\bfseries
Lavori Correlati}]{\fancyplain{}{\bfseries\thepage}}
\lhead[\fancyplain{}{\bfseries\thepage}]{\fancyplain{}{\bfseries
Lavori Correlati}}
%%%%%%%%%%%%%%%%%%%%%%%%%%%%%%%%%%%%%%%%%aggiunge la voce Introduzione
                                        %   nell'indice
\addcontentsline{toc}{chapter}{Lavori Correlati}
%%%%% Questa \`e l'introduzione \cite{K1} \\
In letteratura esistono diversi studi riguardanti i possibili utilizzi dei dati generati dai sensori degli smartphone, molti dei quali cercano di interpretare i dati generati quando il device si trova a bordo di un veicolo per ottenere informazioni di diverso tipo, come per esempio: determinare il tipo di mezzo di trasporto nel quale si trova lo smartphone \cite{K1, K2, K3, K4} e determinare le condizioni della sede stradale o le condizioni di traffico \cite{K5, K6}. \\

I risultati di questi studi sono interessanti per cercare di ottenere un sistema di navigazione inerziale sempre più efficace (INS, Inertial Navigation System).

\section {Inertial Navigation System}
L' Inertial Navigation System (INS) è un sistema in grado di determinare tramite Dead Reckoning la posizione di un oggetto in movimento.\\
Il sistema è formato da una serie di sensori (Inertial Measurement Unit, IMU) e da un calcolatore. I sensori sono, in genere, l'accelerometro, il giroscopio e il magnetometro. \\
Il sistema viene inizializzato con tre parametri: la velocità iniziale, l'orientamento iniziale e le coordinate geografiche di partenza.\\
E' in grado di determinare i cambi di velocità dell'oggetto sfruttando i dati provenienti dall'accelerometro, quindi la variazione di velocità lungo i 3 assi, i cambiamenti di orientamento sfruttando il giroscopio e i cambiamenti di direzione tramite il magnetometro.

E' un sistema ampliamento già usato in ambito marittimo, aeronautico ed aerospaziale. 
In ambito automotive è usato come sistema di navigazione nel caso in cui la tecnologia GPS non sia in grado di funzionare correttamente come nel caso di parcheggi sotterranei, gallerie e in ambito urbano nel caso in cui si percorra una strada in un area piena di edifici che disturbano il passaggio delle onde radio.

\subsection{Dead Reckoning}
Nel calcolo della nuova posizione tramite dead reckoning [Fig 1.1] bisogna tenere conto dei possibili errori cumulativi generati dai sensori ad ogni nuovo ricalcolo.\\ 
La tecnica del map-matching \cite{K7} è in grado di ridurre questi errori, consiste nel associare ad ogni nuova posizione calcolata una posizione reale che corrisponde ad un punto specifico di una strada, si necessità però di una mappa dalla quale ricavare le nuove coordinate.

La posizione del veicolo $(x_k, y_k)$ all'istante $k$ può essere espressa nella seguente maniera \cite{K8}
\begin{center}
$x_k = x_0 + \sum_{i=0}^{k-1} s_i  \cos( \theta_i) $ ,      $y_k = y_0 + \sum_{i=0}^{k-1} s_i  \sin( \theta_i)  $
\end{center}
dove $(x_0, y_0)$ è la posizione iniziale del veicolo al tempo $t_0 $. $s_i$ è la lunghezza del vettore mentre $ \theta_i $ indica il bearing, ossia la direzione del vettore rispetto al nord magnetico.\\
Il bearing relativo è la differenza tra i valori di bearing presi in due istanti consecutivi ed è rappresentata con la lettera $\omega_i $.

Dato il bearing relativo i-esimo $\omega_i$ per ogni istante è facile calcolare $\theta_i$ all'istante $k$:

\begin{figure}[h] 
\centering 
\includegraphics[scale=0.8]{fig1} 
\caption{Dead-Reckoning} 
\end{figure}

\begin{center}
$\theta_k = \sum_{i=0}^k \omega_i$
\end{center}


\section {Sensori}
I sensori disponibili in uno smartphone sono diversi e variano da modello a modello, alcuni sono implementati in maniera hardware, altri in maniera software. Nella piattaforma Android possiamo suddividerli in quattro macroaree \cite{K9}:
\begin{itemize}
\item Motion Sensors: include tutti i sensori che misurano le forze di accelerazione e le forze rotazionali come accelerometro, giroscopio, sensore di gravità. Tutte le forze vengono misurate relativamente ai 3 assi.
\item Environmental Sensors: include tutti i sensori che misurano i parametri ambientali come temperatura, pressione atmosferica, illuminazione, umidità. Quindi: termometro, barometro e sensore di luminosità.
\item Position Sensors: questi sensori determinano la posizione del device nello spazio come il sensore di orientamento, di prossimità e il magnetometro.
\item Location Sensors: sono indispensabili per la geolocalizzazione del device ed hanno un'accuratezza variabile a seconda della tecnologia, delle condizioni atmosferiche e dalla presenza di ostacoli. Abbiamo: GPS e A-GPS(5m - 10m) \cite{K10}; WIFI e Network Position (10m - 35km).
\end{itemize}

\subsection{Accelerometro}
L'accelerometro misura l'accelerazione del device lungo i 3 assi. Nei valori misurati dall'accelerometro è inclusa la forza di gravità. L'unità di misura è $m/s^2$.
Grazie all'accelerometro si possono determinare le variazioni di velocità dell'oggetto in movimento.
\begin{figure}[h] 
\centering 
\includegraphics[scale=0.8]{fig2} 
\caption{Sistema delle coordinate} 
\end{figure}
\subsection{Giroscopio}
Il giroscopio misura la velocità angolare del device intorno ai suoi 3 assi. L'unità di misura è $rad/s$. Con il giroscopio possiamo determinare l'orientamento del device nello spazio.
\subsection{Magnetometro}
Il magnetometro misura l'intensità del campo magnetico relativamente ai suoi 3 assi. L'unità di misura è $\mu T$. I dati misurati dal magnetometro servono a determinare la posizione del device rispetto al nord magnetico.

\subsection{GPS}
Il GPS (Global Positioning System) è un sistema di posizionamento satellitare civile gestito dal governo degli Stati Uniti d'America (GSU). Esso permette di ottenere le coordinate geografiche, quindi la geolocalizzazione, di un oggetto.

Il sistema è formato da tre segmenti: il segmento spaziale, il segmento di controllo ed il segmento utente. I device fanno parte del segmento utente assieme a tutti i dispositivi che fungono da ricevitori GPS.
Nel segmento spaziale abbiamo da 24 a 31 satelliti Navstar.

Il sistema si basa sulla tecnica della trilaterazione. Le coordinate geografiche sono calcolate a partire dal tempo che ci mette il segnale radio a compiere il tragitto device-satellite. 
Ogni satellite periodicamente invia al device la propria posizione e l'istante temporale in cui ha inviato questi dati. Il ricevitore riceve i dati, calcola la differenza di tempo tra l'istante in cui ha ricevuto i dati e l'istante in cui il satellite gli ha inviati e calcola la distanza dal satellite. Il device necessita di almeno 3 satelliti per ottenere la posizione precisa.

\section{Wi-Fi}
Il Wi-Fi è una tecnologia che permette ai dispositivi di trasmettere dati in maniera wireless, quindi senza l'ausilio di cavi. Si basa sulle specifiche dello standard IEEE 802.11.1\\
Oltre a permettere di accedere ad una rete locale è possibile connettersi tramite un access point ad Internet e trasmettere dati a dei server.\\
Una rete Wi-Fi può essere identificata tramite il suo SSID, (Service Set IDentifier). Il SSID consiste in 32 byte che contengono il nome della rete in formato human-readable. Nel caso un device si connettesse ad una determinata rete, l'SSID può essere utile nel geolocalizzarlo. Ad esempio, se un device invia dei dati sfruttando Almawifi possiamo dedurre che sia nei pressi di una struttura universitaria.\\
Il BSSID, invece, è un identificatore univoco, consiste nel MAC address dell'access point. \\
Il RSSI invece indica l'intensità del segnale, dalla quale possiamo dedurre e stimare la distanza del device dall'access point. L'unità di misura è dBm.

La quasi totalità delle reti private ha dei meccanismi di sicurezza che servono a prevenire accessi da parte di device non autorizzati. Abbiamo per esempio WPA (Wireless Protected Access) e WEP (Wired Equivalent Privacy).
Al giorno d'oggi è sempre più facile trovare delle Open-Network \cite{K11}, ossia reti liberi alle quali ci si può collegare gratuitamente. Molti comuni, istituzioni, aree commerciali offrono reti a cui collegarsi liberamente.


\section{OpenStreetMap e Grafi}
OpenStreetMap (OSM) \cite{K12} è un progetto collaborativo open-data che offre un servizio cartografico.\\
OpenStreetMap è stato fondato nel 2004 da Steve Coast e ad ora è usato da molti servizi come Flickr, Strava, WolframAlpha e tanti altri \cite{K13}.

Essendo un progetto open-data è possibile scaricare in locale od usufruire delle API per effettuare query e sviluppare i propri progetti. I dati sono in formato OSM XML.

Gli elementi fondamentali che compongono il \textit{conceptual data model of the physical world} di OpenStreetMap sono:
\begin{itemize}
\item Node: Un nodo consiste in un punto nello spazio, ed è definito da un nodeId univoco, latitudine e longitudine. I nodi possono essere usati per identificare la forma di un edificio o per comporre strade.\\
I nodi possono avere ulteriori tag, come per esempio "highway", "entrance", "natural", etc.\\
I tag sono formati da un insieme di coppie chiave-valore. \\

\begin{figure}[h] 
\centering 
\includegraphics[scale=1]{fig3} 
\caption{Statistiche OpenStreetMap - 2016} 
\end{figure}
\item Way: una way è una lista ordinata di nodi. E' composta da 2 a 2000 nodi. Una way può essere di diversi tipi: chiusa quando l'ultimo nodo è collegato al primo; può essere aperta nel caso di strade; può essere un'area quando delimita una porzione di terreno, se è un area è anche chiusa.\\
Ogni way ha un wayId univoco, una lista di nodi e diversi tag. I più ricorrenti sono "highway", "name", "area", "oneway".
\item Relation: una relation è formata da uno o più tags e da un insieme di nodi/way o altre relazioni che vengono usati per definire relazioni logiche o geografiche tra più elementi. Può essere usata, per esempio, per definire le fermate di un autobus o le stazioni ferroviarie di una determinata linea.
\end{itemize}

\subsection{Grafi}
Un paese, una provincia o addirittura tutto il pianeta può essere inteso come un enorme grafo pesato.\\ 
Ogni intersezione tra due o più ways può essere vista come un nodo del grafo, mentre una way è appunto la strada che collega due intersezioni.
Il peso che diamo ad ogni arco può essere, a seconda degli utilizzi del grafo stesso, o la distanza tra le due intersezioni o il tempo che ci si mette mediamente ad attraversarlo interamente.\\
In questo modo è facile implementare diversi algoritmi di visita partendo da un nodo di partenza.\\
Implementando un grafo tramite liste di adiacenza, una DFS (depth-first search) ha una complessità computazione pari a $\Omega(|V| + |E|) $ dove per $\left|V \right|$ si intende il numero di nodi del grafo e con $\left| E \right|$ il numero di archi.

%%%%%%%%%%%%%%%%%%%%%%%%%%%%%%%%%%%%%%%%%non numera l'ultima pagina sinistra
\clearpage{\pagestyle{empty}\cleardoublepage}



\begin{thebibliography}{90}             %crea l'ambiente bibliografia
\rhead[\fancyplain{}{\bfseries \leftmark}]{\fancyplain{}{\bfseries
\thepage}}
%%%%%%%%%%%%%%%%%%%%%%%%%%%%%%%%%%%%%%%%%aggiunge la voce Bibliografia
                                        %   nell'indice
\addcontentsline{toc}{chapter}{Bibliografia}
%%%%%%%%%%%%%%%%%%%%%%%%%%%%%%%%%%%%%%%%%provare anche questo comando:
%%%%%%%%%%%\addcontentsline{toc}{chapter}{\numberline{}{Bibliografia}}
\bibitem{K1} Widhalm P, Nitsche P, Brandle N. "Transport Mode Detection with Realistic Smartphone Sensor Data". In Proceedings of IEEE CCNC, Las Vegas, USA, 2012; 573-576.
\bibitem{K2} Bedogni L, Di Felice M, Bononi L. "Context-aware Android applications through transportation mode detection techniques". Wirel. Commun. Mob. Comput. 16, 16 (November 2016), 2523-2541.
\bibitem{K3} L. Bedogni, M. Di Felice, L. Bononi, “By Train or By Car? Detecting the User’s Motion Type through Smartphone Sensors Data” on proceedings of the 5th IFIP International Conference Wireless Days 2012 (WD 2012), November 21-23, 2012, Dublin, Ireland
\bibitem{K4} Yu Xiao et al., "Transportation activity analysis using smartphones," 2012 IEEE Consumer Communications and Networking Conference (CCNC), Las Vegas, NV, 2012, pp. 60-61.
\bibitem{K5} R. Bhoraskar, N. Vankadhara, B. Raman and P. Kulkarni, "Wolverine: Traffic and road condition estimation using smartphone sensors," 2012 Fourth International Conference on Communication Systems and Networks (COMSNETS 2012), Bangalore, 2012, pp. 1-6.
\bibitem{K6} Prashanth Mohan, Venkata N. Padmanabhan, and Ramachandran Ramjee. 2008. "Nericell: rich monitoring of road and traffic conditions using mobile smartphones". In Proceedings of the 6th ACM conference on Embedded network sensor systems (SenSys '08). ACM, New York, NY, USA, 323-336.
\bibitem{K7} R. L. French. Map matching origins, approaches and applications. In Second International Symposium on Land Vehicle Navigation, pages 91-116,
Munster, Germany, July 4-7 1989.
\bibitem{K8} Wei-Wen Kao, "Integration of GPS and dead-reckoning navigation systems," Vehicle Navigation and Information Systems Conference, 1991, 1991, pp. 635-643.
\bibitem{K9} \url{https://developer.android.com/guide/topics/sensors/sensors_overview.html}
\bibitem{K10} \url{http://www.gps.gov/systems/gps/performance/accuracy/}
\bibitem{K11} \url{https://openwireless.org}
\bibitem{K12} \url{http://www.openstreetmap.org}
\bibitem{K13} \url{http://wiki.openstreetmap.org/wiki/List_of_OSM-based_services}

\end{thebibliography}

\end{document}
